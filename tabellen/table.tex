\documentclass[a4paper, parskip, 11pt]{scrartcl}

\usepackage[utf8]{inputenc}
\usepackage[T1]{fontenc}
\usepackage[ngerman]{babel}

\usepackage{blindtext}

\begin{document}
%
\blindtext, in folgender Tabelle: 
%
\begin{center}
    \begin{tabular}{l||l|l|l|l|l}
    Name   & Bart & Gürtel & Kapuze & Instrument & Sonstiges \\
    \hline
    Dwalin & blau & gold & dunkelgrün & Bratsche & \\
    Balin & weiß & & purpurrot & Bratsche & \\
    Kili & gelb & silber & blau & Fiedel & Werkzeug \\
    Fili & gelb & silber & blau & Fiedel & Spaten \\
    Dori & & gold & purpurrot & Flöte & \\
    Nori & & gold & purpurrot & Flöte & \\
    Ori & & gold & grau & Flöte & \\
    Oin & & silber & braun & & \\
    Gloin & & silber & blaßgrünsilber & & \\
    Bifur & & & gelb & Klarinette & \\
    Bofur & & & gelb & Klarinette & \\
    Bombur & & & blaßgrün & Trommel & fett \\
    Thorin & & & himmelblau mit & Harfe & sehr berühmt \\
            & & & silberner Schärpe & &
    \end{tabular}
\end{center}
%
\blindtext Besser: siehe Tabelle \ref{tab:zwerge}.
%
% Tabelle als Fließobjekt
\begin{table}
    \centering
    %\captionabove{Die dreizehn Zwerge}
    %\label{tab:zwerge}
    \begin{tabular}{l||l|l|l|l|l}
    Name   & Bart & Gürtel & Kapuze & Instrument & Sonstiges \\
    \hline
    Dwalin & blau & gold & dunkelgrün & Bratsche & \\
    Balin & weiß & & purpurrot & Bratsche & \\
    Kili & gelb & silber & blau & Fiedel & Werkzeug \\
    Fili & gelb & silber & blau & Fiedel & Spaten \\
    Dori & & gold & purpurrot & Flöte & \\
    Nori & & gold & purpurrot & Flöte & \\
    Ori & & gold & grau & Flöte & \\
    Oin & & silber & braun & & \\
    Gloin & & silber & blaßgrünsilber & & \\
    Bifur & & & gelb & Klarinette & \\
    Bofur & & & gelb & Klarinette & \\
    Bombur & & & blaßgrün & Trommel & fett \\
    Thorin & & & himmelblau mit & Harfe & sehr berühmt \\
            & & & silberner Schärpe & &
    \end{tabular}
    \caption{Die dreizehn Zwerge}
    \label{tab:zwerge}
\end{table}
%
\end{document}