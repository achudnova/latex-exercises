% Datum: 31.10.2025
% Name: Alina Chudnova

\documentclass[a4paper]{scrartcl}

% Präambel
\usepackage[utf8]{inputenc} % Eingabekodierung
\usepackage[T1]{fontenc}    % Fontkodierung
\usepackage[ngerman]{babel} % neue deutsche Rechtschreibung
\usepackage{blindtext}

\title{Hausaufgabe 01}
\author{Alina Chudnova, Max Mustermann}
\date{\today}

\begin{document}
\setlength{\parindent}{0pt} % ohne Einrückung
\maketitle                  % Titelseite
\tableofcontents            % Inhaltsverzeichnis
\newpage

% Beginn der Textumgebung
\section{Einleitung mit Blindtext}
\blindtext  % Platzhaltertext


\subsection{Texthervorhebungen}
In diesem Abschnitt werden verschiedene \textbf{Texthervorhebungen} wie \textbf{fett} vorgestellt.

Dieser \emph{Text} ist kursiv.

So sieht die \texttt{Schreibmaschinenschrift} aus.

Hier ist nochmal alles \textsc{großgeschrieben}.


\subsubsection{Kombinierte Textauszeichnungen}
Man kann \textbf{\emph{fett und kursiv}} gleichzeitig verwenden.

Wir können auch \underline{\textsf{unterstreichen und serifenlos gedruckt}} schreiben.

Dieser Text ist \underline{\emph{unterstrichen und in kursiv}}.

Hier sind Textabschnitte, die \textsc{\underline{Kapitälchen und unterstrichen sind.}}


\subsection{Schriftgrößen}
Dieser Text ist {\tiny sehr klein}.

Das ist {\small kleine Schriftgröße}.

Das ist {\large große Schriftgröße}.

{\huge Dieser Abschnitt ist riesig.}


\section{Verschiedene Komibinationen - Schriftgröße und Stil}
{\large\textbf{Dieser Text ist groß und fett.}}

{\small\textsc{Dieser Text ist klein und großgeschrieben gleichzeitig.}}

{\tiny\emph{\underline{Dieser Abschnitt ist winzig, unterstrichen und kursiv.}}}

{\huge\texttt{Riesige Schreibmaschinenschrift!}}

% Ende der Textumgebung
\end{document}
