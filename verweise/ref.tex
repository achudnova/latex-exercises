\documentclass[a4paper, parindent, 11pt]{scrartcl}

\usepackage[utf8]{inputenc}
\usepackage[T1]{fontenc}
\usepackage[ngerman]{babel}

\usepackage{blindtext}

\usepackage{amsmath}
\usepackage{amssymb}

\newcommand{\sectionref}[1]{Abschnitt~\ref{#1}}

\begin{document}
%
\tableofcontents
%
\section{Vorwort}
\label{sec:vorwort}
Hier ist ein Vorwort.
%
%
\section{Sinnloser Abschnitt}
\label{sec:dummy}
Zusammenhänge zwischen Masse, Energe und Beschleunigung werden anhand Gl.~(\ref{eq:einstein})
Gl.~(\ref{eq:newton}) in Abschnitt~\ref{sec:mathe} diskutiert. Hier ein wenig \emph{dummy} Text:

\blindtext
%
\section{Sinnvolle Mathematik}
\label{sec:mathe} % Label-Befehl
%
Nach etwas sinnlosem Text in Abschnitt \sectionref{sec:dummy} und \sectionref{sec:vorwort} hier nun etwas Sinnvolles.
%
\begin{equation}
    \label{eq:einstein}
    E=mc^{2}
\end{equation}
%
\begin{equation}
    \label{eq:newton}
    \vec{F}=m\vec{a}.   
\end{equation}
%
\subsection{Dummy}
\label{subsec:nochmal_dummy}
\blindtext

\end{document}