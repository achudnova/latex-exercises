\documentclass[a4paper, 11pt]{scrartcl}

\usepackage[utf8]{inputenc}
\usepackage[T1]{fontenc}
\usepackage[ngerman]{babel}

\usepackage{graphicx}   % Zum Einbinden von Figuren/Abbildungen nötig
\usepackage{tikz}       % Für TikZ-Grafiken
\usetikzlibrary{arrows}

\begin{document}
%
\section{Hausaufgabe 6.1}
%
Hier steht ein einleitender Text für die erste Aufgabe. Wir betrachten eine kleine Datensammlung.
Wie man in Tabelle \ref{tab:daten} sehr gut erkennen kann, ist Max deutlich jünger als Erika.
%
\begin{table}[h]
    \centering
    \captionabove{Kleine Datenbank}
    \label{tab:daten}
    \begin{tabular}{c|c|c}
        Name & Beruf & Alter \\
        \hline
        Max Mustermann & Physiker & 23 \\
        Erika Mustermann & Grundschullehrerin & 40 
    \end{tabular}
\end{table}
%
\section{Hausaufgabe 6.2}
%
Hier ist etwas Text, in dem ich auf die Abbildungen verweise. Wie in Abbildung \ref{fig:don_knuth} zu sehen ist,
steht dieses Bild alleine. Hingegen besteht die Abbildung \ref{fig:two_pictures} aus zwei Figuren und diese stehen nebeneinander.
Die Gesamtbreite der nebeneinanderstehenden Bilder füllt 90\% der Textbreite aus.
%
\begin{figure}[htb]
    \centering
    \includegraphics[width=0.6\textwidth]{bild1.jpg}
    \caption{Erstes Bild}
    \label{fig:don_knuth}
\end{figure}
%
\begin{figure}[ht]
    \centering
    \includegraphics[width=0.45\textwidth]{bild2.jpg}
    \hfill
    \includegraphics[width=0.45\textwidth]{bild3.png}
    \caption{Zwei Bilder nebeneinander}
    \label{fig:two_pictures}
\end{figure}
%
\clearpage
\section{Hausaufgabe 6.3}
%
In diesem Abschnitt nutzen wir TikZ und Gnuplot. Die Abbildung \ref{fig:gnuplot} zeigt 
mathematische Funktionen, die direkt durch Gnuplot berechnet wurden.

Des Weiteren sehen wir in Abbildung \ref{fig:tikzbeispiel} ein komplexes Diagramm, das rein mit
TikZ erstellt wurde.

Die letzte Abbildung \ref{fig:betragsfunktion} zeigt eine mit Inkscape erstellte Grafik.
Diese Grafik zeigt die Betragsfunktion $y=|x|$.
%
\begin{figure}[ht]
    \centering
    \resizebox{0.9\textwidth}{!}{   % Größe mit resizebox auf 90% der Textbreite anpassen, ! bedeutet Höhe proportional
        \begin{tikzpicture}[domain=0:4]
            \draw[very thin,color=gray] (-0.1,-1.1) grid (3.9,3.9);
            \draw[->] (-0.2,0) -- (4.2,0) node[right] {$x$};
            \draw[->] (0,-1.2) -- (0,4.2) node[above] {$f(x)$};
            \draw[color=red] plot[id=x] function{x} 
                node[right] {$f(x) =x$};
            \draw[color=blue] plot[id=sin] function{sin(x)} 
                node[right] {$f(x) = \sin x$};
            \draw[color=orange] plot[id=exp] function{0.05*exp(x)} 
                node[right] {$f(x) = \frac{1}{20} \mathrm e^x$};
        \end{tikzpicture}
    }
    \caption{GNUPLOT basics}
    \label{fig:gnuplot}
\end{figure}
%
%
% Beispiel von der Seite: https://texample.net/tikz/examples/
% Author: Rasmus Pank Roulund
\begin{figure}[ht]
    \centering
    \resizebox{0.9\textwidth}{!}{   % Größe auf 90% der Textbreite anpassen
        \begin{tikzpicture}
            [
            scale=5,
            axis/.style={very thick, ->, >=stealth'},
            important line/.style={thick},
            dashed line/.style={dashed, thin},
            pile/.style={thick, ->, >=stealth', shorten <=2pt, shorten
            >=2pt},
            every node/.style={color=black}
            ]
            % axis
            \draw[axis] (-0.1,0)  -- (1.1,0) node(xline)[right]
                {$G\uparrow/T\downarrow$};
            \draw[axis] (0,-0.1) -- (0,1.1) node(yline)[above] {$E$};
            % Lines
            \draw[important line] (.15,.15) coordinate (A) -- (.85,.85)
                coordinate (B) node[right, text width=5em] {$Y^O$};
            \draw[important line] (.15,.85) coordinate (C) -- (.85,.15)
                coordinate (D) node[right, text width=5em] {$\mathit{NX}=x$};
            % Intersection of lines
            \fill[red] (intersection cs:
            first line={(A) -- (B)},
            second line={(C) -- (D)}) coordinate (E) circle (.4pt)
            node[above,] {$A$};
            % The E point is placed more or less randomly
            \fill[red]  (E) +(-.075cm,-.2cm) coordinate (out) circle (.4pt)
                node[below left] {$B$};
            % Line connecting out and ext balances
            \draw [pile] (out) -- (intersection of A--B and out--[shift={(0:1pt)}]out)
                coordinate (extbal);
            \fill[red] (extbal) circle (.4pt) node[above] {$C$};
            % line connecting  out and int balances
            \draw [pile] (out) -- (intersection of C--D and out--[shift={(0:1pt)}]out)
                coordinate (intbal);
            \fill[red] (intbal) circle (.4pt) node[above] {$D$};
            % line between out og all balanced out :)
            \draw[pile] (out) -- (E);
        \end{tikzpicture}
    }

    \caption{Beispiel von der Seite \color{blue}https://texample.net/tikz/examples/}
    \label{fig:tikzbeispiel}
\end{figure}
%
% Meine Vektorgrafik, erstellt mit Hilfe von Inkscape
\begin{figure}[ht]
    \centering
    \def\svgwidth{0.9\textwidth}       % Größe auf 90% der Textbreite anpassen
    \input{bild4.pdf_tex}
    \caption{Betragsfunktion}
    \label{fig:betragsfunktion}
\end{figure}
%
\end{document}