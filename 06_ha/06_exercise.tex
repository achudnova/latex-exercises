\documentclass[a4paper, 11pt]{scrartcl}

\usepackage[utf8]{inputenc}
\usepackage[T1]{fontenc}
\usepackage[ngerman]{babel}

\usepackage{graphicx}
\usepackage{tikz}

\begin{document}
%
\section{Hausaufgabe 6.1}
%
\begin{table}[h]
    \centering
    \captionabove{Kleine Datenbank}
    \label{tab:daten}
    \begin{tabular}{c|c|c}
        Name & Beruf & Alter \\
        \hline
        Max Mustermann & Physiker & 23 \\
        Erika Mustermann & Grundschullehrerin & 40 
    \end{tabular}
\end{table}
%
\section{Hausaufgabe 6.2}
%
Hier ist etwas Text, in dem ich auf die Abbildungen verweise. Wie in Abbildung \ref{fig:don_knuth} zu sehen ist,
steht dieses Bild alleine. Hingegen besteht die Abbildung \ref{fig:two_pictures} aus zwei Figuren und diese stehen nebeneinander.
%
\begin{figure}[htb]
    \centering
    \includegraphics[width=0.7\textwidth]{bild1.jpg}
    \caption{Erstes Bild}
    \label{fig:don_knuth}
\end{figure}
%
\begin{figure}[ht]
    \centering
    \includegraphics[width=0.45\textwidth]{bild2.jpg}
    \hfill
    \includegraphics[width=0.45\textwidth]{bild3.png}
    \caption{Zwei Bilder nebeneinander}
    \label{fig:two_pictures}
\end{figure}
%
\clearpage
\section{Hausaufgabe 6.3}
%
\begin{figure}[ht]
    \centering
    \begin{tikzpicture}[domain=0:4]
        \draw[very thin,color=gray] (-0.1,-1.1) grid (3.9,3.9);
        \draw[->] (-0.2,0) -- (4.2,0) node[right] {$x$};
        \draw[->] (0,-1.2) -- (0,4.2) node[above] {$f(x)$};
        \draw[color=red] plot[id=x] function{x} 
            node[right] {$f(x) =x$};
        \draw[color=blue] plot[id=sin] function{sin(x)} 
            node[right] {$f(x) = \sin x$};
        \draw[color=orange] plot[id=exp] function{0.05*exp(x)} 
            node[right] {$f(x) = \frac{1}{20} \mathrm e^x$};
    \end{tikzpicture}
\end{figure}

%
\end{document}