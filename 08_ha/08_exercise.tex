\documentclass{article}

\usepackage[utf8]{inputenc}
\usepackage[T1]{fontenc}
\usepackage[ngerman]{babel}

\usepackage{blindtext}
\usepackage{layout}
\usepackage[a4paper, left=2cm, right=3.5cm, top=3cm]{geometry}
\usepackage{pdfpages}
\usepackage{fancyhdr}
\pagestyle{fancy}
%
% Kopfzeile
\lhead{\slshape LaTeX}
\chead{Alina Chudnova}
\rhead{\slshape Hausaufgabe 8}
%
% Fußzeile
\rfoot{\scshape TU Berlin}
\cfoot{\thepage}
%
\usepackage{hyperref}
\hypersetup{
    pdftitle = {LaTeX Hausaufgabe},
    pdfauthor = {Alina Chudnova},
    pdfkeywords = {latex, hyperref, referenzen},
    pdfdisplaydoctitle = true,
    colorlinks = true,
    urlcolor = blue, 
    filecolor = red,
    linkcolor = cyan
}
%
\begin{document}
%
\tableofcontents
%
\section{Einleitung}
\label{sec:einleitung}
Dies ist die Einleitung.
In Abschnitt \ref{sec:grundlagen} werden die Grundlagen erklärt.
Eine externe Quelle findet sich bei Wikipedia\footnote{\href{https://de.wikipedia.org/wiki/LaTeXRL}{LaTeX Wikipedia}}.
%
\newpage
%
\section{Hausaufgabe 8.1}
\layout % Seitenlayout anzeigen
%
\newpage
\blindtext[5]
%
\newgeometry{left=5cm, right=1cm, top=2cm, bottom=4cm, footnotesep=1.5cm} % Abstand Text - Fußnoten
Hier ist ein Satz mit einer Fußnote\footnote{Dies ist die erste Fußnote: \href{https://www.overleaf.com/}{Overleaf-Website}}.
\blindtext[5]
%
\restoregeometry
\blindtext[3]
%
\section{Hausaufgabe 8.2}
\label{sec:aufgabe82}
%
\includepdf[landscape=true, nup=2x1, pages={2, 1}]{latex_ueb_08.pdf}
%
\section{Hausaufgabe 8.3}
\label{sec:grundlagen}
%
Hier werden Grundlagen behandelt. Ein weiterer Verweis führt zu Abschnitt \ref{sec:einleitung}.
Wie bereits in Abschnitt \ref{sec:aufgabe82} gezeigt, lassen sich PDFs direkt einbinden.
Siehe die eingebundene Datei \href{run:latex_ueb_08.pdf}{(PDF-Datei)}.
Siehe offizielle TU Website: \href{https://www.tu-berlin.de}{TU Berlin}
%
\end{document}