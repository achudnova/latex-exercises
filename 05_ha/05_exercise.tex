\documentclass[a4paper, 11pt]{scrartcl}

\usepackage[utf8]{inputenc}
\usepackage[T1]{fontenc}
\usepackage[ngerman]{babel}

\usepackage{amsthm} % für Theorem-Umgebungen (Definitionen, Beweise)
\usepackage{mathtools}
\usepackage{cancel}
% \usepackage{amsmath}
% \usepackage{amssymb}

%\newtheorem{Name}{Ausgabe}
\newtheorem{deff}{Definition}[section]
\newtheorem{bem}{Bemerkung}
\newtheorem{bsp}[bem]{Beispiel}

%
\begin{document}
%
\section{Hausaufgabe 5.1}
%

\begin{deff}
Eine Abbildung $f$ von $D$ nach $W$ ordnet jedem Element $x \in D$ genau ein Element $f(x) \in W$ zu. 
\end{deff}

\begin{bsp}
Dies ist ein Beispieltext für das erste Beispiel.
\end{bsp}

\section{Gemeinsamer Zähler}

\begin{bsp}
Hier ist ein Beispiel.
\end{bsp}

\begin{deff}
Hier wird noch etwas definiert.
\end{deff}

\begin{bem}
Hier ist eine kleine Bemerkung.
\end{bem}

\section{Testabschnitt}

\begin{bem}
Der Zähler für Bemerkungen läuft weiter.
\end{bem}

\begin{deff}
Neue Section, neue Nummer, neue Definition.
\end{deff}

\end{document}