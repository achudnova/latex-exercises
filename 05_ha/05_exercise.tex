\documentclass[a4paper, 11pt]{scrartcl}

\usepackage[utf8]{inputenc}
\usepackage[T1]{fontenc}
\usepackage[ngerman]{babel}

\usepackage{amsthm} % für Theorem-Umgebungen (Definitionen, Beweise)
\usepackage{cancel}
\usepackage{mathtools}
% \usepackage{amsmath}
\usepackage{amssymb}

%\newtheorem{Name}{Ausgabe}
\newtheorem{deff}{Definition}[section]
\newtheorem{bem}{Bemerkung}
\newtheorem{bsp}[bem]{Beispiel}

\begin{document}
%
\section{Hausaufgabe 5.1}
%
%
\begin{deff}
Eine Abbildung $f$ von $D$ nach $W$ ordnet jedem Element $x \in D$ genau ein Element $f(x) \in W$ zu. 
\end{deff}
%
\begin{bsp}
Dies ist ein Beispieltext für das erste Beispiel.
\end{bsp}
%
\subsection{Gemeinsamer Zähler}
%
\begin{bsp}
Hier ist ein Beispiel.
\end{bsp}
%
\begin{deff}
Hier wird noch etwas definiert.
\end{deff}
%
\begin{bem}
Hier ist eine kleine Bemerkung.
\end{bem}
%
\subsection{Testabschnitt}
%
\begin{bem}
Der Zähler für Bemerkungen läuft weiter.
\end{bem}
%
\begin{deff}
Neue Section, neue Nummer, neue Definition.
\end{deff}
%
\section{Hausaufgabe 5.2}
%
Cancel-Paket:
\[
    \cancel{a} + b = \xcancel{-a} + b
\]
%
\[
    \begin{pmatrix*}[c]
        1 & 2 & 3 \\
        10 & 20 & 30 \\
        100 & 200 & 300
    \end{pmatrix*}
    =
    \begin{pmatrix*}[l]
        1 & 2 & 3 \\
        10 & 20 & 30 \\
        100 & 200 & 300   
    \end{pmatrix*}
    =
    \begin{pmatrix*}[r]
        1 & 2 & 3 \\
        10 & 20 & 30 \\
        100 & 200 & 300
    \end{pmatrix*}
\]
%
Seien $n \geq 2, \alpha_{1} \dots , \alpha_n \in \mathbb{R}$ und
\[
    A_{n} := 
    \begin{bmatrix}
        1 & \alpha_{1} & \alpha_{1}^{2} & \dots & \alpha_{1}^{n-1} \\
        1 & \alpha_{2} & \alpha_{2}^{2} & \dots & \alpha_{2}^{n-1} \\
        \vdots & \vdots & \vdots & & \vdots \\
        1 & \alpha_{n} & \alpha_{n}^{2} & \dots & \alpha_{n}^{n-1}
    \end{bmatrix}
\]
%
Zeige folgende Formel:
\begin{align*}
    \det(A_{n}) &= \prod_{1 \le i < j \le n} (\alpha_{j}-\alpha_{i})
    \intertext{sieht hässlich aus im Vergleich zu}
    \det(A_{n}) &= \prod_{\mathclap{1 \le i < j \le n}} (\alpha_{j}-\alpha_{i})
\end{align*}
%
\section{Hausaufgabe 5.3}
%
\section{Einführung}
\label{sec:einfuehrung}
%
In diesem Abschnitt üben wir, Referenzen/Verweise richtig zu setzen.
Im Abschnitt~\ref{sec:formeln} werden wir uns ein paar grundlegende mathematische Formeln~\eqref{eq:pqformel} und~\eqref{eq:pythagoras} anschauen.
Insbesondere schauen wir uns eine wichtige Formel~\eqref{eq:dimensionsformel} im Abschnitt~\ref{subsec:linalgebra} an.
Dieser Abschnitt befindet sich auf Seite~\pageref{subsec:linalgebra}.
%
\section{Mathematische Formeln}
\label{sec:formeln}
%
\begin{equation}
    \label{eq:pythagoras}
    a^{2} + b^{2} = c^{2}
\end{equation}
%
\begin{equation}
    \label{eq:pqformel}
    x_{1,2} = -\frac{p}{2} \pm \sqrt{\left(\frac{p}{2}\right)^{2}-q}
\end{equation}
%
%
\newpage
\subsection{Lineare Algebra}
\label{subsec:linalgebra}
%
Dimensionsformel:
\begin{equation}
    \label{eq:dimensionsformel}
    \dim(V) = \dim(\mathrm{Bild}_{f}) + \dim(\mathrm{Kern}_{f})
\end{equation}
%
\begin{equation}
    \label{eq:eigenwert}
    A \cdot v = \lambda \cdot v
\end{equation}
%
\end{document}