\documentclass[a4paper, 11pt, bibtotoc]{scrartcl}
% Literaturverzeichnis im Inhaltsverzeichnis: bibtotoc

\usepackage[utf8]{inputenc}
\usepackage[T1]{fontenc}
\usepackage[ngerman]{babel}
\usepackage{csquotes}
%\usepackage[babel, german=quotes]{csquotes}
\usepackage{array}
\usepackage[
    backend=biber,              % behandelt utf8-encoding korrekt
    style=alphabetic]{biblatex} % Zitierstil (numeric)
\addbibresource{literatur.bib}  % im selben Verzeichnis - Literaturdatenbank

\usepackage{graphicx}

\begin{document}
%
\section{Hausaufgabe 7.1}
%
\begin{table}[htbp]
    \resizebox{\textwidth}{!}{ % Breite fest, Höhe automatisch
    \begin{tabular}{c p{5cm} p{3.5cm} p{3cm} c}
        \hline
        \textbf{Software-Name}  & \textbf{Charakteristika}     & \textbf{Betriebssystem}    & \textbf{Kosten}      & \textbf{Präferenz} \\
        \hline
        Zotero                  
        & Lokal und webbasiert, schneller Import von Literaturdaten,
        Notizen anlegen, Zusammenarbeit mit LaTeX-Editoren 
        & Windows, Mac, Linux, Web-Version    
        & Lokal: konstenlos, Web: bis 300 MB kostenlos  
        & 1 \\
        \hline
        Citavi                  
        & Lokal und webbasiert, einfache Handhabung, 
        komplexe Möglichkeiten, Zusammenarbeit mit LaTeX-Editoren 
        & nur Windows, Web-Version 
        & kostenfrei für TU-Angehörige 
        & 3 \\
        \hline    
        JabRef                  
        & Lokal und webbasiert, Zusammenarbeit mit LaTeX-Editoren, 
        Prioritätsangaben möglich, nur über gemeinsame SQL-DB
        & Windows, Mac, Linux, Web-Version 
        & konstenfrei 
        & 2 \\
        \hline
    \end{tabular}}
\end{table}
% Hausaufgabe 7.2
% Wählt eine Literaturverwaltungs-Software aus und erstellt damit eine Literatur- ¨
% Sammlung, die mindestens 10 Bucher, 10 wissenschaftliche Aufs ¨ atze (Artikel) ¨
% und 10 Quellen anderer Art (darunter mindestens 5 Webseiten) enthalt. ¨
% • Erstellt mit der gewahlten Literaturverwaltungs-Software ein ¨ .bib-File. Das
% File muss im gleichen Ordner gespeichert werden wie das .tex-File, damit ich
% es kompilieren kann!

% Hausaufgabe 7.3
% • Schreibt etwas Text und zitiert dabei mindestens 15 Quellen aus eurer LiteraturSammlung, auf keine Fall aber alle.
% • Unter den Quellen sollen sich Bucher, Artikel und Webseiten befinden. ¨
% • Erstellt am Ende des Dokumentes das zugehorige Literaturverzeichnis. Der ¨
% verwendete Zitierstil ist freigestellt.
\section{Hausaufgabe 7.3}
%
Text hier \cite{testkey}
%
\section{Literaturverzeichnis}
%
\printbibliography
% Das ist ein Beispieltext mit einer Zitation \cite{testkey}.

% \cite[Präfix][Suffix]{Kürzel}
% \parencite[Präfix][Suffix]{Kürzel}
% \footcite[Präfix][Suffix]{Kürzel}
% \textcite[Präfix][Suffix]{Kürzel}
% \printbibliography
%
\end{document}