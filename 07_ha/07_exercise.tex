\documentclass[a4paper, 11pt]{scrartcl}
% Literaturverzeichnis im Inhaltsverzeichnis: bibtotoc

\usepackage[utf8]{inputenc}
\usepackage[T1]{fontenc}
\usepackage[ngerman]{babel}
\usepackage{csquotes}

\usepackage[
    backend=biber,              % behandelt utf8-encoding korrekt
    style=numeric]{biblatex}    % Zitierstil
\addbibresource{literatur.bib}  % im selben Verzeichnis - Literaturdatenbank

\usepackage{graphicx}   % Tabelle auf Textbreite skalieren (resizebox)

\begin{document}
%
\section{Hausaufgabe 7.1}
%
\begin{table}[htbp]
    \resizebox{\textwidth}{!}{ % Breite fest, Höhe automatisch
    \begin{tabular}{c p{5cm} p{3.5cm} p{3cm} c}
        \hline
        \textbf{Software-Name}  & \textbf{Charakteristika}     & \textbf{Betriebssystem}    & \textbf{Kosten}      & \textbf{Präferenz} \\
        \hline
        Zotero                  
        & Lokal und webbasiert, schneller Import von Literaturdaten,
        Notizen anlegen, Zusammenarbeit mit LaTeX-Editoren 
        & Windows, Mac, Linux, Web-Version    
        & Lokal: konstenlos, Web: bis 300 MB kostenlos  
        & 1 \\
        \hline
        Citavi                  
        & Lokal und webbasiert, einfache Handhabung, 
        komplexe Möglichkeiten, Zusammenarbeit mit LaTeX-Editoren 
        & nur Windows, Web-Version 
        & kostenfrei für TU-Angehörige 
        & 3 \\
        \hline    
        JabRef                  
        & Lokal und webbasiert, Zusammenarbeit mit LaTeX-Editoren, 
        Prioritätsangaben möglich, nur über gemeinsame SQL-DB
        & Windows, Mac, Linux, Web-Version 
        & konstenfrei 
        & 2 \\
        \hline
    \end{tabular}}
\end{table}
%
% Hausaufgabe 7.2
%
\section{Hausaufgabe 7.3}
%
Die Meteorologie hat sich im Laufe des 20. Jahrhunderts zu einer hochtechnologischen Wissenschaft gewandelt, die auf Computerberechnungen angewiesen ist \cite{nebeker_calculating, balzer_wettervorhersage}. 

Während grundlegende Begriffe in klassischen Nachschlagewerken definiert sind \cite{schirmer_meyers, noauthor_meteorologie}, 
steht die Forschung heute vor einem erneuten Paradigmenwechsel durch den Einsatz Künstlicher Intelligenz \cite{speiser_artificial}.

Aktuelle Übersichtsstudien zeigen, dass ML-Methoden in der modernen Klimavorhersage zunehmend an Bedeutung gewinnen \cite{zhang_machine, chen_machine}. 
ML-Algorithmen werden als zentrale Modellkomponenten eingesetzt \cite{de_burgh-day_machine}. 
Auch Methoden des Supervised Structure Learning finden Anwendung \cite{friston_supervised}.

Die Einsatzgebiete sind dabei vielfältig. Sie reichen von der Analyse verschiedener Wetterereignisse wie Stürmen \cite{mcgovern_review} über spezifische Simulationen \cite{fu_simulation} bis hin zu Analysen von Mars-Wetterdaten \cite{pant_machine_2023}. 
In der Materialforschung, zur Vorhersage von Strahlungskühlungseffekten, kommen diese Verfahren zum Einsatz \cite{shi_machine}.

Trotz der Erfolge, die auch von Tech-Giganten wie Google vorangetrieben werden \cite{noauthor_weathernext}, bleibt die Nachvollziehbarkeit der Ergebnisse eine Herausforderung.
Die Forschung konzentriert sich daher verstärkt auf interpretierbare KI-Modelle, um die "Black Box"-Problematik zu lösen \cite{yang_interpretable}. 
Die Black Box Problematik wird auch in aktuellen Fachartikeln und Lexika diskutiert \cite{eichmann_wetterlexikon, noauthor_ki_klima}.

\footnotesize \textit{Anmerkung: Dieser Text wurde mithilfe von Künstlicher Intelligenz (Gemini) sprachlich korrigiert und überarbeitet.}
%
\section{Literaturverzeichnis}
\printbibliography
%
\end{document}