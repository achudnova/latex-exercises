\documentclass[a4paper, 11pt]{scrartcl}

\usepackage[utf8]{inputenc}
\usepackage[T1]{fontenc}
\usepackage[ngerman]{babel}
\usepackage{csquotes}
%\usepackage[babel, german=quotes]{csquotes}
\usepackage{array}
\usepackage[
    backend=biber,
    style=alphabetic]{biblatex}
\addbibresource{literatur.bib}

\usepackage{graphicx}

\begin{document}
%
\section{Hausaufgabe 7.1}
%
\begin{table}[htbp]
    \resizebox{\textwidth}{!}{ % Breite fest, Höhe automatisch
    \begin{tabular}{c p{5cm} p{3.5cm} p{3cm} c}
        \hline
        \textbf{Software-Name}  & \textbf{Charakteristika}     & \textbf{Betriebssystem}    & \textbf{Kosten}      & \textbf{Präferenz} \\
        \hline
        Zotero                  
        & Lokal und webbasiert, schneller Import von Literaturdaten,
        Notizen anlegen, Zusammenarbeit mit LaTeX-Editoren 
        & Windows, Mac, Linux, Web-Version    
        & Lokal: konstenlos, Web: bis 300 MB kostenlos  
        & 1 \\
        \hline
        Citavi                  
        & Lokal und webbasiert, einfache Handhabung, 
        komplexe Möglichkeiten, Zusammenarbeit mit LaTeX-Editoren 
        & nur Windows, Web-Version 
        & kostenfrei für TU-Angehörige 
        & 3 \\
        \hline    
        JabRef                  
        & Lokal und webbasiert, Zusammenarbeit mit LaTeX-Editoren, 
        Prioritätsangaben möglich, nur über gemeinsame SQL-DB
        & Windows, Mac, Linux, Web-Version 
        & konstenfrei 
        & 2 \\
        \hline
    \end{tabular}}
\end{table}
%
% Das ist ein Beispieltext mit einer Zitation \cite{testkey}.

% \cite[Präfix][Suffix]{Kürzel}
% \parencite[Präfix][Suffix]{Kürzel}
% \footcite[Präfix][Suffix]{Kürzel}
% \textcite[Präfix][Suffix]{Kürzel}
% \printbibliography
%
\end{document}