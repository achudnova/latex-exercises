\documentclass{scrlttr2}

\LoadLetterOption{privatbrief-1}

\begin{document}

% Datum
\setkomavar{date}{\today}

% Betreff
\setkomavar{subject}{Bewerbung als studentische Hilfskraft in der Abteilung Cybersecurity}

% Definition des Serienbrief-Befehls
% #1 = Name/Firma
% #2 = Name/Abteilung
% #3 = Straße
% #4 = PLZ
% #5 = Ort
\newcommand\sbrief[5]{
    \begin{letter}{#1\\#2\\#3\\#4 #5}
    \opening{Sehr geehrte Damen und Herren,}

    Hiermit bewerbe ich mich auf Ihre Stellenausschreibung von der Hochschulwebseite als 
    "studentische Hilfskraft für XY (m/w/d)". 

    Ich studiere bereits seit Anfang des letzten Jahres an der TU Berlin und plane nach
    meinem Bachelor-Studium auch mein Master-Studium im Bereich Informatik zu absolvieren.
    Die ausgeschriebene Position als studentische Hilfskraft in Ihrer Abteilung hat sofort
    mein Interesse geweckt, da ich meine bisherigen Kenntnisse ideal mit den Anforderungen
    der Stelle verbinden kann.

    Während des Studiums habe ich mir ein fundiertes Wissen in den Bereichen XY angeeignet.
    Meine Stärken liegen in der analytischen Denkweise, meiner Fähigkeit, komplexe Probleme
    strukturiert anzugehen, und meiner ausgeprägten Kommunikationsfähigkeit.

    Dafür stehe ich ab sofort zur Verfügung. Ich würde mich sehr darüber freuen, von Ihnen
    zu hören und Sie persönlich kennenzulernen. 

    \closing{Mit freundlichen Grüßen}

    \end{letter}
}

% Adressdatei laden
\sbrief{Musterfirma GmbH}{z.H. Herrn Müller}{Industriestraße 1}{12345}{Musterstadt}
\sbrief{StartUp Berlin}{Personalabteilung}{Torstraße 10}{10119}{Berlin}
\sbrief{IT-Systeme Meyer}{Frau Schmidt}{Gartenweg 5}{20095}{Hamburg}
\sbrief{Cybersec Solutions}{Recruiting Team}{Hafenplatz 1}{50667}{Köln}
\sbrief{Behörde für Digitales}{Referat 42}{Amtsgasse 9}{80331}{München}

\end{document}