\documentclass[a4paper, parskip, 11pt]{scrartcl}

\usepackage[utf8]{inputenc}
\usepackage[T1]{fontenc}
\usepackage[ngerman]{babel}

\usepackage{amsmath}
\usepackage{amssymb}

% Neue Befehle für die Zahlenmengen
\newcommand{\N}{\mathbb{N}}
\newcommand{\Z}{\mathbb{Z}}
\newcommand{\R}{\mathbb{R}}
\newcommand{\C}{\mathbb{C}}

% weitere Befehle definieren
\newcommand{\tb}{\textbackslash}
\newcommand{\klammer}[1]{\left(#1 \right)}
\newcommand{\intervall}[2]{\left[#1, #2 \right]}

\begin{document}

\section{Hausaufgabe 4.1}

Zu den wichtigsten Zahlenbereichen in der Mathematik gehören die natürlichen Zahlen $\N$ (z.B. $0,1,2,3\dots$), 
die ganzen Zahlen $\Z$ (z.B. $\dots,-3,-3,-1,0,\dots$), die reelen Zahlen $\R$ (z.B. $\sqrt{5}$)
und die komplexen Zahlen $\C$ (z.B. $1+\mathrm{i}$).

Hier teste ich den \texttt{\tb klammer}-Befehl:
\[
    x \cdot \klammer{\frac{1}{2} + 2 + \cos(x)}
\]

Hier test ich den \texttt{\tb intervall}-Befehl mit zwei Argumenten:
\[
    x \in \intervall{0}{10}
\]
%
\end{document}
