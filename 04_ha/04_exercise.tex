\documentclass[a4paper, parskip, 11pt]{scrartcl}

\usepackage[utf8]{inputenc}
\usepackage[T1]{fontenc}
\usepackage[ngerman]{babel}

\usepackage{amsmath}
\usepackage{amssymb}

% Neue Befehle für die Zahlenmengen
\newcommand{\N}{\mathbb{N}}
\newcommand{\Z}{\mathbb{Z}}
\newcommand{\R}{\mathbb{R}}
\newcommand{\C}{\mathbb{C}}

% weitere Befehle definieren
\newcommand{\tb}{\textbackslash}
\newcommand{\klammer}[1]{\left(#1 \right)}
\newcommand{\intervall}[2]{\left[#1, #2 \right]}
\newcommand{\gklammer}[1]{\left\{#1 \right\}}

\begin{document}
%
\section{Hausaufgabe 4.1}
%
Zu den wichtigsten Zahlenbereichen in der Mathematik gehören die natürlichen Zahlen $\N$ (z.B. $0,1,2,3\dots$), 
die ganzen Zahlen $\Z$ (z.B. $\dots,-3,-2,-1,0,\dots$), die reellen Zahlen $\R$ (z.B. $\sqrt{5}$)
und die komplexen Zahlen $\C$ (z.B. $1+\mathrm{i}$).

Hier teste ich den \texttt{\tb klammer}-Befehl:
\[
    x \cdot \klammer{\frac{1}{2} + 2 + \cos(x)}
\]

Hier test ich den \texttt{\tb intervall}-Befehl mit zwei Argumenten:
\[
    x \in \intervall{0}{10}
\]
Das bedeutet, $x$ liegt zwischen $0$ und $10$.
%
\section{Hausaufgabe 4.2}
%
Es gilt:
\begin{align*}
    M_2             &= \gklammer{\klammer{x,y}\in\R^{2} \mid x^{2} > y} \\
    \mathring{M}_2  &= \gklammer{\klammer{x,y}\in\R^{2} \mid x^{2} > y} = M_2 \\
    \partial M_2    &= \gklammer{\klammer{x,y}\in\R^{2} \mid x^{2} = y} \\
    \bar{M}_2       &= \gklammer{\klammer{x,y}\in\R^{2} \mid x^{2} \ge y}
\end{align*}
%
Wir betrachten die auf $\R^{2}$ definierte Abbildung
\[
    f(x,y) :=
    \left\{
    \begin{array}{cl}
        \frac{x^{9}+3x^{2}y^{7}}{\klammer{x^{2}+y^{2}}^{2}} & \text{ für } \klammer{x,y} \neq \klammer{0,0} \\[0.5cm]
        0 & \text{ für } \klammer{x,y} = 0
    \end{array}
    \right.
\]
%
Mit Hilfe des \texttt{\tb cfrac}-Befehls setzen wir einen schönen Kettenbruch:
\[
    \sqrt{2} = 1 + \cfrac{1}{2+ \cfrac{1}{2+ \cfrac{1}{2+ \cfrac{1}{2+ \cfrac{1}{\ddots}}}}}
\]
%
\section{Hausaufgabe 4.3}
%
Wir betrachten das folgende lineare Gleichungssystem:
\[
    \begin{pmatrix}
        -1 & 2 & 1 \\
        3 & -8 & -2 \\
        1 & 0 & 4 
    \end{pmatrix}
    \begin{pmatrix}
        x_1 \\ x_2 \\ x_3 \\
    \end{pmatrix}
    = \begin{pmatrix}
        -2 \\ 4 \\ -2
    \end{pmatrix}
\]
Wir lösen es mit Hilfe des Gauß-Algorithmus:
\begin{align*}
    \begin{matrix}
        -1 & 2 & 1 \mid 1 \\
        3 & -8 & -2 \mid -2 \\
        1 & 0 & 4 \mid 4
    \end{matrix} \quad
    &\rightsquigarrow
    \begin{matrix}
        -1 & 2 & 1 \mid -2 \\
        0 & -2 & 1 \mid -2 \\
        0 & 2 & 5 \mid -4
    \end{matrix} \\[0.5cm]
    &\rightsquigarrow
    \begin{matrix}
         -1 & 2 & 1 \mid -2 \\
         0 & -2 & 1 \mid -2 \\
         0 & 0 & 6 \mid -6
    \end{matrix}
\end{align*}
%
% In den folgenden Gleichungen sollen B, E, j fett gesetzt werden
\begin{gather*}
    \nabla \cdot \mathbf{B} = 0 \\
    \nabla \cdot \mathbf{E} = \frac{\rho}{\varepsilon_{0}} \\
    \nabla \times \mathbf{B} = \mu_{0} \mathbf{j} + \mu_{0}\varepsilon_{0} \frac{\partial \mathbf{E}}{\partial t} \\
    \nabla \times \mathbf{E} = -\frac{\partial \mathbf{B}}{\partial t} \\
    \frac{\partial^{2} \mathbf{E}}{\partial x^{2}} + \frac{\partial^{2} \mathbf{E}}{\partial y^{2}} + 
    \frac{\partial^{2} \mathbf{E}}{\partial z^{2}} - \frac{1}{c^{2}} \frac{\partial^{2} \mathbf{E}}{\partial t^{2}} = 0
\end{gather*}
%
Wir machen folgenden Ansatz für eine partikuläre Lösung:
\[
    y_p(x) = x(A_0 \cos(x) + B_0 \sin(x))e^{x}
\]
%
Für die erste Ableitung erhalten wir:
\begin{equation*}
    \begin{split}
        y'_p(x) &= (A_0 \cos(x) + B_0 \sin(x))e^{x} \\
                &+ x(-A_0 \sin(x) + B_0 \cos(x))e^{x} \\
                &+ x(A_0 \cos(x) + B_0 \sin(x))e^{x} \\
                &= e^{x}(\cos(x)(A_0 + A_0x + B_0x) + \sin(x)(B_0-A_0x+B_0x))           
    \end{split}
\end{equation*}
%
Für die zweite Ableitung erhalten wir:
\begin{equation*}
    \begin{split}
        y''_p(x) &= e^{x}(\cos(x)(A_0 + A_0x + B_0x) + \sin(x)(B_0 - A_0x + B_0x)) \\
                 &-\sin(x)(A_0 + A_0x + B_0x) + \cos(x)(A_0 + B_0) \\
                 &+\cos(x)(B_0 - A_0x + B_0x) + \sin(x)(-A_0x + B_0) \\
                 &= e^{x} \cos(x)(A_0 + A_0x + B_0x + A_0 + B_0 + B_0 - A_0 + B_0x) \\
                 &+ \sin(x)(B_0 - A_0x + B_0x - A_0 - A_0x - B_0x - A_0 + B_0) \\
                 &= e^{x}(\cos(x)(2A_0 + 2B_0 + 2B_0x) \\
                 &+ \sin(x)(-2A_0 + 2B_0 - 2A_0x))
    \end{split}
\end{equation*}
%
\end{document}
