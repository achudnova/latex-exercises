% Hausaufgabe 3.1
\documentclass[a4paper, 11pt]{scrartcl}

\usepackage[utf8]{inputenc}
\usepackage[T1]{fontenc}
\usepackage[ngerman]{babel}

% Pakete für mathematischen Formelsatz
\usepackage{amsmath}
\usepackage{amssymb}

\title{03 Hausaufgabe}
\author{Alina Chudnova}
\date{\today}

\begin{document}
\maketitle
%
% Hausaufgabe 3.2
% Aufgabe 1
Es gilt:
\begin{align*}
    \lim_{k\to\infty}\left(\sqrt{k+1}-\sqrt{k}\right) 
    & = \lim_{k\to\infty}\left(\frac{\sqrt{k+1}-\sqrt{k}}{\sqrt{k+1}+\sqrt{k}}\cdot\left(\sqrt{k+1}+\sqrt{k}\right)\right) \\
    & = \lim_{k\to\infty}\left(\frac{(k+1)-k}{\sqrt{k+1}+\sqrt{k}}\right) \\                                                  
    & = \lim_{k\to\infty}\left(\frac{1}{\sqrt{k+1}+\sqrt{k}}\right) \\
    & = 0
\end{align*}
%
% Aufgabe 2
Wir betrachten die Folge $(x_n,y_n)_{n\in N} := \left(\frac{1}{n^{2}},\frac{1}{n}\right)_{n\in N}$. Sie erfüllt:
\begin{equation}
    \lim_{n\to\infty}\left(x_n,y_n\right) = \left(0,0\right)
\end{equation}
Weiter gilt:
\begin{align}
    g\left(x_n,y_n\right) & = g\left(\frac{1}{n^{2}},\frac{1}{n}\right) \\
                          & = \frac{\frac{1}{n^{2}}\cdot\left(\frac{1}{n}\right)^{2}}{\left(\frac{1}{n^{2}}\right)^{2}+\left(\frac{1}{n}\right)^{4}} \\
                          & = \frac{\frac{1}{n^{4}}}{2\cdot\frac{1}{n^{4}}} \\
                          & = \frac{1}{2} \\
                          & \neq 0 = g(0,0)
\end{align}
%
% Aufgabe 3
Mit der Betrachtung des Grenzwertes $l\to\infty$ erhalten wir:
\begin{equation}
    \Vert \exp(A)-\exp(B) \Vert_{2} \leq \Vert A-B \Vert_{2}\cdot \exp(\max\left\{\Vert A \Vert_{2},\Vert B \Vert_{2}\right\})\tag{TKS}
\end{equation}
%
\newpage
% Hausaufgabe 3.3
% Aufgabe 1
Für $a,b > 0$ sei $R := [0,a] \times [0,b]$ und $f(x,y) := xe^{xy}.$ Wir berechnen das Integral:
\begin{gather*}
    \int_1 f(x,y) \, dV = \int_{0}^{a} \left(\int_{0}^{b}xe^{xy} \, dy\right) \, dx \\
                        = \int_{0}^{a} \left(x\cdot\left[\frac{1}{x}e^{xy}\right]_{y=0}^{y=b}\right) \, dx \\
                        = \int_{0}^{a} \left((e^{xb}-1)\right) \, dx \\
                        = \left[\frac{1}{b} e^{xb}-x\right]_{0}^{a} \\
                        = \frac{1}{b} (e^{ab}-1)-a
\end{gather*}
%
% Aufgabe 2
Wir betrachten die folgende Differentialgleichung:
\[
    y' = e^{y} \cdot \sin(x)
\] 
Wir können sie mit der Methode der Trennung der Variablen lösen. Zunächst gilt:
\begin{align*}
    y' & = e^{y} \sin(x) \\
    \Leftrightarrow e^{-y}y' & = \sin(x) \\
    \Leftrightarrow e^{-y} \frac{dy}{dx} & = \sin(x) \\
    \Leftrightarrow e^{-y} dy & = \sin(x) dx
\end{align*}
%
% Aufgabe 3
Seien $n,l \in \mathrm{N}$ und $A,B \in \mathrm{R}^{n \times n}$.

Dann gilt:
\begin{gather*}
    \sum_{k=0}^{l} \frac{A^{k}}{k!} - \sum_{k=0}^{l} \frac{B^{k}}{k!} = \sum_{k=0}^{l} \frac{1}{k!} \left( A^{k}-B^{k} \right) \\
    = \sum_{k=1}^{l} \frac{1}{k!} \left( A^{k}-B^{k} \right) \\
    = \dots \\
    = (A-B) \sum_{k=0}^{l-1} \frac{1}{(k+1)!} \sum_{j=0}^{k} A^{j} B^{k-j}
\end{gather*}
%
\end{document}