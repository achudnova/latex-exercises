\documentclass[11pt, a4paper]{scrartcl}

\usepackage[T1]{fontenc}
\usepackage[utf8]{inputenc}
\usepackage[ngerman]{babel}

\usepackage{csquotes}

\usepackage{amsmath}
\usepackage{amssymb}

% Newcommand: Definitionen
% 
% Abkürzungen für 'lange Zeichenketten'
\newcommand{\tb}{\textbackslash}

% Kommandos für den Mathematikmodus
\newcommand{\imag}{\mathrm{i}}
\newcommand{\euler}{\mathrm{e}}
\newcommand{\abs}[1]{\left|#1\right|}  % [1] -> 1 Argument; #1 Variablenargument wird hier eingesetzt
\newcommand{\deriv}[2]{\frac{\mathrm{d} #1}{\mathrm{d} #2}}
%
\begin{document}
\setlength{\parindent}{0pt}
%
\section{\texttt{newcommand}-Befehle}
%
Meine Notizen sind unter \texttt{C:\textbackslash Benutzer\textbackslash Alina\textbackslash Documents}.

Meine Notizen sind unter \texttt{C:\tb Benutzer\tb Alina\tb Documents}.
%
\[
    \mathrm{e}^{\mathrm{i}\varphi}=\cos(\varphi)+\mathrm{i}\sin(\varphi)
    \quad \euler^{\imag\varphi}=\cos(\varphi)+\imag\sin(\varphi)
\]
%
\[
    \left|a\right| \quad
    \left|\frac{a}{b}\right| \quad
    \abs{a} \quad
    \abs{\frac{a}{b}}
\]
%
\[
    \frac{\mathrm{d}\sin(x)}{\mathrm{d}x}=\cos(x) \quad
    \deriv{\sin(x)}{x}=\cos(x) \quad
    \deriv{}{x}\sin(x)=\cos(x)
\]
\end{document}