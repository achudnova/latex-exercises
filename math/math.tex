\documentclass[a4paper, 12pt]{scrartcl}

\usepackage[utf8]{inputenc}
\usepackage[T1]{fontenc}
\usepackage[ngerman]{babel}

\usepackage{amsmath}
\usepackage{amssymb}


\begin{document}
\setlength{\parindent}{0pt}
%
\section{Mathematischer Formelsatz}
Seien \begin{math}
    x\in\mathbb{R}
\end{math} und $y\in\mathbb{R}$ dann gilt
%
\begin{equation}
    |x+y| \leq |x| + |y|
    \label{formel1}
\end{equation}
Formel (\ref{formel1}) wird Dreiecksungleichung genannt. \\
%
Auch mehrzeilige Gleichungen sind möglich:
%
\begin{align}
    (a+b)^2 & = (a+b)(a+b) \\
            & = a^2+2ab+b^2
\end{align}
Es gibt auch abgesetzte Formeln ohne Nummern:
%
\begin{equation*}
    \int_{0}^{\pi/4}\sin(x)\,\mathrm{d}x=1-\frac{1}{\sqrt{2}}
    \int_{0}^{\pi/4}f(x)\,\mathrm{d}x = 1-\frac{1}{\sqrt{2}}
\end{equation*}
Für abgesetzte Formeln ohne Nummern gibt es auch eine Kurzform:
\[
    R=\frac{\sum_{i=1}^{n}(x_i-\bar{x})(y_i-\bar{y})}
    {\left[\sum_{i=1}^{n}(x_i-\bar{x}^2)\sum_{i=1}^{n}(y_i-\bar{y}^2)\right]^{1/2}}
\]
%
\begin{equation*}
    A_{m,n} =
    \begin{pmatrix}
        1 & 2 & 3 \\
        4 & 5 & 6 \\
        7 & 8 & 9 \\
    \end{pmatrix}
\end{equation*}

\newpage
\section{Tabellen-Übung}
Hier noch eine Tabelle:
\begin{center}
    \begin{tabular}{c c c}
        A & B & C \\
        \hline
        1 & 2 & 3 \\
        4 & 5 & 6 \\
        7 & 8 & 9 \\
    \end{tabular}
\end{center}


\end{document}