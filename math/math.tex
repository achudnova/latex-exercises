\documentclass[11pt, a4paper]{scrartcl}

\usepackage[utf8]{inputenc}
\usepackage[T1]{fontenc}
\usepackage[ngerman]{babel}

\usepackage{amsmath}
\usepackage{amssymb}

\usepackage{siunitx} % zum Setzen von Einheiten
\sisetup{locale = DE}

\begin{document}
\setlength{\parindent}{0pt}
%
\section{Mathematischer Formelsatz}
Seien \begin{math}
    x\in\mathbb{R}
\end{math} und $y\in\mathbb{R}$ dann gilt
%
\begin{equation}
    |x+y| \leq |x| + |y|
    \label{formel1}
\end{equation}
Formel (\ref{formel1}) wird Dreiecksungleichung genannt. \\
%
Auch mehrzeilige Gleichungen sind möglich:
%
\begin{align}
    (a+b)^2 & = (a+b)(a+b) \\
            & = a^2+2ab+b^2
\end{align}
Es gibt auch abgesetzte Formeln ohne Nummern:
%
\begin{equation*}
    \int_{0}^{\pi/4}\sin(x)\,\mathrm{d}x=1-\frac{1}{\sqrt{2}}
    \int_{0}^{\pi/4}f(x)\,\mathrm{d}x = 1-\frac{1}{\sqrt{2}}
\end{equation*}
Für abgesetzte Formeln ohne Nummern gibt es auch eine Kurzform:
\[
    R=\frac{\sum_{i=1}^{n}(x_i-\bar{x})(y_i-\bar{y})}
    {\left[\sum_{i=1}^{n}(x_i-\bar{x}^2)\sum_{i=1}^{n}(y_i-\bar{y}^2)\right]^{1/2}}
\]
%
\begin{equation*}
    A_{m,n} =
    \begin{pmatrix}
        1 & 2 & 3 \\
        4 & 5 & 6 \\
        7 & 8 & 9 \\
    \end{pmatrix}
\end{equation*}

\newpage
\section{Tabellen-Übung}
Hier noch eine Tabelle:
\begin{center}
    \begin{tabular}{c c c}
        A & B & C \\
        \hline
        1 & 2 & 3 \\
        4 & 5 & 6 \\
        7 & 8 & 9 \\
    \end{tabular}
\end{center}
%
%
\section{Typische Fehler beim Formelsatz}
\textbf{Falscher Formelsatz:} Mit der Variable x gilt:
\[
    sin(x)=0 \quad mit \quad x=n\pi, n\in\mathbb{Z}.
\]
\textbf{Korrekter Formelsatz:} Mit der Variable $x$ gilt:
\[
    \sin(x)=0 \text{ mit } x= n\pi, n\in\mathbb{Z}.
\]
\textbf{Falscher Formelsatz:} 
\[
    e^{i\varphi}=\cos(\varphi)+i\sin(\varphi)
\]
\textbf{Korrekter Formelsatz:}
\[
    \mathrm{e}^{\mathrm{i}\varphi}=\cos(\varphi)+\mathrm{i}\sin(\varphi)
\]
\textbf{Falscher Formelsatz:}
\[
    \left(\frac{d^{2}}{dr^{2}}+\frac{1}{r}\frac{d}{dr}\right)\psi(r)=h(r)
\]
\textbf{Korrekter Formelsatz:}
\[
    \left(\frac{\mathrm{d}^{2}}{\mathrm{d}r^{2}}+\frac{1}{r}\frac{\mathrm{d}}{\mathrm{d}r}\right)\psi(r)=h(r)
\]
%
%
\section{Zahlen und Einheiten im Mathemathikmodus}
%
\begin{center}
    \begin{tabular}{ll}
        $l=3.0e03 km$ & Zahl aus Programm; Einheit nicht aufrecht \\
        $l=3.0\cdot 10^{3} km$ & Einheit nicht aufrecht \\
        $l=3.0\cdot 10^{3} \mathrm{km}$ & korrekt? \\
        $l=\SI{3.0e03}{\kilo\meter}$ & richtig! \\
    \end{tabular}
\end{center}
Die Frequenz betrug nur $f=1.0 \mathrm{khz}$ anstatt der 
erwarteten $f=1.0 \mathrm{mhz}$.

Die Frequenz betrug nur $f=\SI{1.0}{\kilo\hertz}$ anstatt der
erwarteten $f=\SI{1,0}{\mega\hertz}$.
\end{document}