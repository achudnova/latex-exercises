\documentclass[11pt]{scrartcl}

\usepackage[utf8]{inputenc}
\usepackage[T1]{fontenc}
\usepackage[ngerman]{babel}
\usepackage{blindtext}

% Einrückung und Absatzabstand
% \usepackage{parskip}

% Zeilenabstand: \usepackage{setspace}
% \onehalfspacing \singlespacing \doublespacing
% \begin{singlespace} .. \end{singlespace}
% \begin{spacing}{3} .. \end{spacing}

% Textausrichtung: \usepackage{ragged2e}

% Optional: Layout-Paket zum Anzeigen der Seitenmaße
% \usepackage{layout}

% \usepackage{needspace}
% \usepackage{fancyhdr}

%\usepackage[a4paper, left=3cm, top=2cm]{geometry}
% Standard für das komplette Dokument festlegen
\usepackage[left=2cm, right=2cm, top=2cm, bottom=2cm]{geometry}

\usepackage{pdfpages} % Einbinden von PDF Dateien

% \usepackage{hyperref}

\begin{document}

% ===== Titelseite =====
\begin{titlepage}
    {\Huge\bfseries Mein Dokumenttitel\par}

    {\large Technische Universität Berlin\par}
\end{titlepage}

% Seitenlayout ausgeben
% \layout

% ===== Abschnitt mit geändertem Layout =====
\newgeometry{left=3cm, right=4cm, top=1cm, bottom=2.5cm}
\newpage
\blindtext

% ===== Zurück zum Standard-Layout =====
\restoregeometry
\blindtext

% alle Seiten: \includepdf[pages=-]{filename}
% von x bis y: \includepdf[pages={x-y}]{filename}
% querkant: \includepdf[landscape=true, pages=-]{filename}
% 2 auf 1: \includepdf[pages={x-y}, nup=2x1]{filename}
% Rahmen: frame=true
% 

\end{document}