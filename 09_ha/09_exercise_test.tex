\documentclass{beamer}

% Style
\mode<presentation>{
    \usetheme{Boadilla}
    \usecolortheme{beaver}
}

\usepackage[utf8]{inputenc}
\usepackage[T1]{fontenc}
\usepackage[ngerman]{babel}
\usepackage{graphicx} % Bilder
\usepackage{booktabs} % Tabellen
\usepackage{blindtext}

% Einstellungen für die Präsentation
\title[test]{Erste Präsentation}
\author{Alina Chudnova}
\institute[TUB]{Technische Universität Berlin}
\date{\today}

\begin{document}

% Titelseite, 1 frame = 1 slide
\begin{frame}
\titlepage
\end{frame}

%Inhaltsvereichnis
\begin{frame}
\frametitle{Inhalt}
\tableofcontents
\end{frame}

\section{Grundlagen}
\subsection{Merkmale}

\begin{frame}
\frametitle{Merkmale}
\blindtext
\end{frame}

\begin{frame}
\frametitle{Merkmale als Stichpunkte}
\begin{itemize}
    \item erster Punkt \pause
    \item zweiter Punkt
\end{itemize}
\end{frame}

\subsection{Äußeres}

\begin{frame}
\frametitle{Äußeres in Textblöcken}
\begin{block}{XY}
Text Text Text
\end{block}

\begin{block}{Test}
Text Text Text
\end{block}
\end{frame}

\begin{frame}
\frametitle{Mehrere Spalten}
\begin{columns}[c] % c = vertikal zentriert
\column{.4\textwidth} % linke Spalte
\textbf{Eigenschaften}
\begin{enumerate}
    \item Punkt
    \item wichtiger Punkt
\end{enumerate}

\column{.5\textwidth} % rechte Spalte
Etwas Text
\end{columns}
\end{frame}

\begin{frame}
\frametitle{title}
\end{frame}

\end{document}