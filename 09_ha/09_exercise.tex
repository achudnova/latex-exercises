\documentclass{beamer}

% Style
\mode<presentation>{
    \usetheme{Boadilla}
    \usecolortheme{beaver}
}

\usepackage[utf8]{inputenc}
\usepackage[T1]{fontenc}
\usepackage[ngerman]{babel}
\usepackage{graphicx} % Bilder
\usepackage{booktabs} % Tabellen
\usepackage{blindtext}

% Einstellungen für die Präsentation
\title{Hausaufgabe 9}
\author{Alina Chudnova}
\institute[TUB]{Technische Universität Berlin}
\date{\today}

\begin{document}

% Titelseite, 1 frame = 1 slide
\begin{frame}
\titlepage
\end{frame}

%Inhaltsvereichnis
\begin{frame}
\frametitle{Giederung}
\tableofcontents
\end{frame}

\section{Einführung}
\begin{frame}
\frametitle{Einführung}
%
\begin{itemize}
    \item ein wichtiger Stichpunkt
    \item eine Folie pro Section
\end{itemize}
\end{frame}

\section{Eigenschaften}
\begin{frame}
\frametitle{Eigenschaften - Beispiele}
\blindtext
\end{frame}

\section{Beispiel-Block}
\begin{frame}
\frametitle{Beispiel-Block in \LaTeX}
%
\begin{block}{Definition}
    Ein Vektor v ist ein Element eines Vektorraums.
\end{block}
%
\begin{block}{Beispiel}
    Bei einem Nullvektor sind alle Elemente 0. Seine Länge (Betrag) ist null.
\end{block}
\end{frame}

%
\end{document}