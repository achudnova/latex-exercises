\documentclass[a4paper, 12pt]{scrartcl}

\usepackage[utf8]{inputenc}
\usepackage[T1]{fontenc}
\usepackage[ngerman]{babel}

\title{Hausaufgabe 02}
\author{Alina Chudnova}
\date{\today}

\begin{document}
\maketitle
%
% Hausaufgabe 2.2
\begin{description}
    \item[Ein Stichpunkt] Inhalt zum ersten Stichpunkt.
    \item[Zweiter Stichpunkt] Noch mehr Inhalt.  
\end{description}
%
\begin{itemize}
    \item Text in der ersten Ebene
    \begin{itemize}
        \item ein wenig Text in der zweiten Ebene
        \item Noch ein weiterer Unterpunkt
    \end{itemize}
    \item Zweiter Punkt
\end{itemize}
%
\begin{enumerate}
    \item erstes Item in der ersten Ebene
    \begin{enumerate}
        \item Item in der zweiten Ebene
        \begin{enumerate}
            \item Item in der dritten Ebene
            \begin{enumerate}
                \item Item in der vierten und damit der letzten Ebene
            \end{enumerate}
        \end{enumerate}
    \end{enumerate}
\end{enumerate}
% Die maximale Verschachtelungstiefe für eine Listen-Umgebung beträgt 4 Ebenen.
% Durch das Mischen von itemize und enumerate Umgebungen kann aber eine Tiefe von sechs Ebenen erreicht werden. 
% Das liegt daran, dass Latex itemize und enumerate-Ebenen getrennt voneinander zählt.
\begin{itemize}
    \item Stichpunkt - 1. Ebene
    \begin{enumerate}
        \item Nummerierter Stichpunkt - 2. Ebene
        \begin{itemize}
            \item Noch ein Stichpunkt - 3. Ebene
            \begin{enumerate}
                \item Noch ein Nummerierter Stichpunkt - 4. Ebene
                \begin{itemize}
                    \item Stichpunkt - 5. Ebene
                    \begin{enumerate}
                        \item Nummerierter Stichpunkt drei in der 6. Ebene
                    \end{enumerate}
                \end{itemize}
            \end{enumerate}
        \end{itemize}
    \end{enumerate}
\end{itemize}
%
% Hausaufgabe 2.3
\begin{center}
    \begin{tabular}{l l c r r}
        \multicolumn{3}{c}{multiple column} & column2 & column3 \\ \hline
        1 & 1 & 1 & 1 & 1 \\
        2 & 2 & 2 & 2 & 2 \\
        3 & 3 & 3 & 3 & 3 \\
        4 & 4 & 4 & 4 & 4 \\
    \end{tabular}
\end{center}
%
\begin{center}
    \begin{tabular}{|*{6}{r|} *{6}{l|}}
        i1 & i2 & i3 & i4 & i5 & i6 & 
        i7 & i8 & i9 & i10 & i11 & i12 \\
        i13 & i14 & i15 & i16 & i17 &
        i18 & i19 & i20 & i21 & i22 & i23 & i24 \\
        i25 & i26 & i27 & i28 & i29 & i30 &
        i31 & i32 & i33 & i34 & i35 & i36 \\
    \end{tabular}
\end{center}
%
\begin{center}
    \begin{tabular}{|p{5cm}|c|c|}
        text1 & text2 & text3 \\ \hline
        Dieser Text ist sehr lang und wird automatisch auf mehrere Zeilen verteilt. & text2 & text3 \\
        text4 & text5 & text6 \\
        text7 & text8 & text9 \\
    \end{tabular}
\end{center}
\end{document}